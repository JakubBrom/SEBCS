%% Generated by Sphinx.
\def\sphinxdocclass{report}
\documentclass[letterpaper,10pt,english]{sphinxmanual}
\ifdefined\pdfpxdimen
   \let\sphinxpxdimen\pdfpxdimen\else\newdimen\sphinxpxdimen
\fi \sphinxpxdimen=.75bp\relax
\ifdefined\pdfimageresolution
    \pdfimageresolution= \numexpr \dimexpr1in\relax/\sphinxpxdimen\relax
\fi
%% let collapsible pdf bookmarks panel have high depth per default
\PassOptionsToPackage{bookmarksdepth=5}{hyperref}

\PassOptionsToPackage{warn}{textcomp}
\usepackage[utf8]{inputenc}
\ifdefined\DeclareUnicodeCharacter
% support both utf8 and utf8x syntaxes
  \ifdefined\DeclareUnicodeCharacterAsOptional
    \def\sphinxDUC#1{\DeclareUnicodeCharacter{"#1}}
  \else
    \let\sphinxDUC\DeclareUnicodeCharacter
  \fi
  \sphinxDUC{00A0}{\nobreakspace}
  \sphinxDUC{2500}{\sphinxunichar{2500}}
  \sphinxDUC{2502}{\sphinxunichar{2502}}
  \sphinxDUC{2514}{\sphinxunichar{2514}}
  \sphinxDUC{251C}{\sphinxunichar{251C}}
  \sphinxDUC{2572}{\textbackslash}
\fi
\usepackage{cmap}
\usepackage[T1]{fontenc}
\usepackage{amsmath,amssymb,amstext}
\usepackage{babel}



\usepackage{tgtermes}
\usepackage{tgheros}
\renewcommand{\ttdefault}{txtt}



\usepackage[Bjarne]{fncychap}
\usepackage[,numfigreset=3,mathnumfig]{sphinx}

\fvset{fontsize=auto}
\usepackage{geometry}


% Include hyperref last.
\usepackage{hyperref}
% Fix anchor placement for figures with captions.
\usepackage{hypcap}% it must be loaded after hyperref.
% Set up styles of URL: it should be placed after hyperref.
\urlstyle{same}


\usepackage{sphinxmessages}
\setcounter{tocdepth}{2}


\usepackage{mathtools}
\usepackage{inputenc}
\usepackage{siunitx}


\title{SEBCS Documentation}
\date{Feb 24, 2022}
\release{2.1.1}
\author{Jakub Brom}
\newcommand{\sphinxlogo}{\vbox{}}
\renewcommand{\releasename}{Release}
\makeindex
\begin{document}

\ifdefined\shorthandoff
  \ifnum\catcode`\=\string=\active\shorthandoff{=}\fi
  \ifnum\catcode`\"=\active\shorthandoff{"}\fi
\fi

\pagestyle{empty}
\sphinxmaketitle
\pagestyle{plain}
\sphinxtableofcontents
\pagestyle{normal}
\phantomsection\label{\detokenize{index::doc}}



\chapter{Introduction}
\label{\detokenize{index:introduction}}
\sphinxAtStartPar
The SEBCS for QGIS is software enabling calculation of energy balance and crop water stress features (heat fluxes, evaporative fraction, Bowen ratio, Omega factor, CWSI etc.) from Landsat satellite data (L5 TM, L7 ETM+, L8 OLI/TIRS, L9 OLI/TIRS) and also from other devices (e.g. UAV).
The calculation procedure uses an approach based on Penman\sphinxhyphen{}Monteith method, SEBAL method and gradient approach of the energy balance characteristics calculation.

\begin{DUlineblock}{0em}
\item[] 
\end{DUlineblock}
\begin{description}
\item[{Author:}] \leavevmode
\begin{DUlineblock}{0em}
\item[] 
\item[] doc. Ing. Jakub Brom, Ph.D.
\end{DUlineblock}

\end{description}

\begin{DUlineblock}{0em}
\item[] 
\end{DUlineblock}

\sphinxAtStartPar
© 2020, Jihočeská univerzita v Českých Budějovicích, Zemědělská fakulta

\noindent\sphinxincludegraphics[width=8cm]{{ZF_JU_RGB_Positive}.png}


\chapter{User tutorial}
\label{\detokenize{index:user-tutorial}}

\section{Tutorial}
\label{\detokenize{tutorial:tutorial}}\label{\detokenize{tutorial::doc}}
\sphinxAtStartPar
TODO


\chapter{Calculation}
\label{\detokenize{index:calculation}}

\section{Calculation}
\label{\detokenize{calculation:calculation}}\label{\detokenize{calculation::doc}}
\sphinxAtStartPar
TODO


\chapter{Abbreviation}
\label{\detokenize{index:abbreviation}}

\section{Abbreviation list}
\label{\detokenize{abbrev:abbreviation-list}}\label{\detokenize{abbrev::doc}}

\begin{savenotes}\sphinxatlongtablestart\begin{longtable}[c]{|\X{15}{80}|\X{15}{80}|\X{50}{80}|}
\sphinxthelongtablecaptionisattop
\caption{Abbreviations used within the SEBCS documentation\strut}\label{\detokenize{abbrev:id3}}\\*[\sphinxlongtablecapskipadjust]
\hline
\sphinxstyletheadfamily 
\sphinxAtStartPar
Feature
&\sphinxstyletheadfamily 
\sphinxAtStartPar
Unit
&\sphinxstyletheadfamily 
\sphinxAtStartPar
Description
\\
\hline
\endfirsthead

\multicolumn{3}{c}%
{\makebox[0pt]{\sphinxtablecontinued{\tablename\ \thetable{} \textendash{} continued from previous page}}}\\
\hline
\sphinxstyletheadfamily 
\sphinxAtStartPar
Feature
&\sphinxstyletheadfamily 
\sphinxAtStartPar
Unit
&\sphinxstyletheadfamily 
\sphinxAtStartPar
Description
\\
\hline
\endhead

\hline
\multicolumn{3}{r}{\makebox[0pt][r]{\sphinxtablecontinued{continues on next page}}}\\
\endfoot

\endlastfoot

\sphinxAtStartPar
\(a_w\)
&
\sphinxAtStartPar
°
&
\sphinxAtStartPar
Aspect
\\
\hline
\sphinxAtStartPar
\(B\)
&
\sphinxAtStartPar
rel.
&
\sphinxAtStartPar
Spectral bands, spectral reflectance
\\
\hline
\sphinxAtStartPar
\(C\)
&
\sphinxAtStartPar
const.
&
\sphinxAtStartPar
Constant
\\
\hline
\sphinxAtStartPar
\(c_p\)
&
\sphinxAtStartPar
\(J.kg^{-1}.K^{-1}\)
&
\sphinxAtStartPar
Thermal heat capacity of dry air (cp = 1012 \(J.kg^{-1}.K^{-1}\))
\\
\hline
\sphinxAtStartPar
\(d\)
&
\sphinxAtStartPar
m
&
\sphinxAtStartPar
Effective height of the canopy
\\
\hline
\sphinxAtStartPar
\(DMT\)
&
\sphinxAtStartPar
m
&
\sphinxAtStartPar
Digital model of terrain
\\
\hline
\sphinxAtStartPar
\(d{\epsilon}\)
&
\sphinxAtStartPar
unitless
&
\sphinxAtStartPar
Effect of natural surfaces geometry distribution and their internal reflection
\\
\hline
\sphinxAtStartPar
\(E_a\)
&
\sphinxAtStartPar
kPa
&
\sphinxAtStartPar
Water vapour pressure of saturated air
\\
\hline
\sphinxAtStartPar
\(e_a\)
&
\sphinxAtStartPar
kPa
&
\sphinxAtStartPar
Water vapour pressure
\\
\hline
\sphinxAtStartPar
\(E_s\)
&
\sphinxAtStartPar
kPa
&
\sphinxAtStartPar
Water vapour pressure of saturated air at the surface
\\
\hline
\sphinxAtStartPar
\(e_s\)
&
\sphinxAtStartPar
kPa
&
\sphinxAtStartPar
Water vapour pressure at the surface
\\
\hline
\sphinxAtStartPar
\(F_L\)
&
\sphinxAtStartPar
unitless
&
\sphinxAtStartPar
Factor of Monin\sphinxhyphen{}Obukhov length
\\
\hline
\sphinxAtStartPar
\(g\)
&
\sphinxAtStartPar
\(m.s^{-2}\)
&
\sphinxAtStartPar
Gravitational forcing
\\
\hline
\sphinxAtStartPar
\(G\)
&
\sphinxAtStartPar
\(W.m^{-2}\)
&
\sphinxAtStartPar
Ground heat flux
\\
\hline
\sphinxAtStartPar
\(h\)
&
\sphinxAtStartPar
m
&
\sphinxAtStartPar
Canopy height
\\
\hline
\sphinxAtStartPar
\(H\)
&
\sphinxAtStartPar
\(W.m^{-2}\)
&
\sphinxAtStartPar
Sensible heat flux
\\
\hline
\sphinxAtStartPar
\(H_s\)
&
\sphinxAtStartPar
°
&
\sphinxAtStartPar
Hour angle
\\
\hline
\sphinxAtStartPar
\(h_{st}\)
&
\sphinxAtStartPar
m
&
\sphinxAtStartPar
Mean height of canopy around the meteo\sphinxhyphen{}station
\\
\hline
\sphinxAtStartPar
\(I_s\)
&
\sphinxAtStartPar
\(W.m^{-2}\)
&
\sphinxAtStartPar
Incident shortwave solar radiation perpendicular to beam
\\
\hline
\sphinxAtStartPar
\(L\)
&
\sphinxAtStartPar
m
&
\sphinxAtStartPar
Monin\sphinxhyphen{}Obukhov length
\\
\hline
\sphinxAtStartPar
\(L_{dry}\)
&
\sphinxAtStartPar
m
&
\sphinxAtStartPar
Monin\sphinxhyphen{}Obukhow length for dry air
\\
\hline
\sphinxAtStartPar
\(Lat\)
&
\sphinxAtStartPar
°
&
\sphinxAtStartPar
Latitude
\\
\hline
\sphinxAtStartPar
\(Long\)
&
\sphinxAtStartPar
°
&
\sphinxAtStartPar
Longitude
\\
\hline
\sphinxAtStartPar
\(N\)
&
\sphinxAtStartPar
unitless
&
\sphinxAtStartPar
No. of day in year
\\
\hline
\sphinxAtStartPar
\(MSAVI\)
&
\sphinxAtStartPar
unitless
&
\sphinxAtStartPar
Modified Soil Adjusted Vegetation Index
\\
\hline
\sphinxAtStartPar
\(NDMI\)
&
\sphinxAtStartPar
unitless
&
\sphinxAtStartPar
Normalized Difference Moisture Index
\\
\hline
\sphinxAtStartPar
\(NDVI\)
&
\sphinxAtStartPar
unitless
&
\sphinxAtStartPar
Normalized Difference Vegetation Index
\\
\hline
\sphinxAtStartPar
\(P\)
&
\sphinxAtStartPar
kPa
&
\sphinxAtStartPar
Atmospheric pressure
\\
\hline
\sphinxAtStartPar
\(Pv\)
&
\sphinxAtStartPar
unitless
&
\sphinxAtStartPar
Fractional vegetation index
\\
\hline
\sphinxAtStartPar
\(r_a\)
&
\sphinxAtStartPar
\(s.m^{-1}\)
&
\sphinxAtStartPar
Surface aerodynamic resistance for heat and momentum transfer
\\
\hline
\sphinxAtStartPar
\(r_c\)
&
\sphinxAtStartPar
\(s.m^{-1}\)
&
\sphinxAtStartPar
Surface resistance for water vapour transfer
\\
\hline
\sphinxAtStartPar
\(r_{cp}\)
&
\sphinxAtStartPar
\(s.m^{-1}\)
&
\sphinxAtStartPar
Surface aerodynamic resistance for water vapour transfer in case of potential evapotranspiration
\\
\hline
\sphinxAtStartPar
\(Rh\)
&
\sphinxAtStartPar
\%
&
\sphinxAtStartPar
Relative humidity of air
\\
\hline
\sphinxAtStartPar
\(Rl_{\uparrow}\)
&
\sphinxAtStartPar
\(W.m^{-2}\)
&
\sphinxAtStartPar
Upward longwave radiation flux
\\
\hline
\sphinxAtStartPar
\(Rl_{\downarrow}\)
&
\sphinxAtStartPar
\(W.m^{-2}\)
&
\sphinxAtStartPar
Downward longwave radiation flux
\\
\hline
\sphinxAtStartPar
\(Rn\)
&
\sphinxAtStartPar
\(W.m^{-2}\)
&
\sphinxAtStartPar
Total net radiation
\\
\hline
\sphinxAtStartPar
\(R_{blue}\)
&
\sphinxAtStartPar
\(rel.\)
&
\sphinxAtStartPar
Spectral reflectance in blue spectral area (BLUE band)
\\
\hline
\sphinxAtStartPar
\(R_{green}\)
&
\sphinxAtStartPar
\(rel.\)
&
\sphinxAtStartPar
Spectral reflectance in green spectral area (GREEN band)
\\
\hline
\sphinxAtStartPar
\(R_{red}\)
&
\sphinxAtStartPar
\(rel.\)
&
\sphinxAtStartPar
Spectral reflectance in red spectral area (RED band)
\\
\hline
\sphinxAtStartPar
\(R_{nir}\)
&
\sphinxAtStartPar
\(rel.\)
&
\sphinxAtStartPar
Spectral reflectance in NIR spectral area (NIR band)
\\
\hline
\sphinxAtStartPar
\(R_{swir1}\)
&
\sphinxAtStartPar
\(rel.\)
&
\sphinxAtStartPar
Spectral reflectance in SWIR spectral area at approx. 1.6 \({\mu}m\) (SWIR1 band)
\\
\hline
\sphinxAtStartPar
\(R_{swir2}\)
&
\sphinxAtStartPar
\(rel.\)
&
\sphinxAtStartPar
Spectral reflectance in NIR spectral area at approx. 2.2 \({\mu}m\) (SWIR2 band)
\\
\hline
\sphinxAtStartPar
\(Rs_{\uparrow}\)
&
\sphinxAtStartPar
\(W.m^{-2}\)
&
\sphinxAtStartPar
Reflected shortwave radiation flux
\\
\hline
\sphinxAtStartPar
\(Rs_{\downarrow}\)
&
\sphinxAtStartPar
\(W.m^{-2}\)
&
\sphinxAtStartPar
Incomming shortwave radiation flux
\\
\hline
\sphinxAtStartPar
\(Rs_{\downarrow const}\)
&
\sphinxAtStartPar
\(W.m^{-2}\)
&
\sphinxAtStartPar
Incomming shortwave radiation flux at the horizontal surface
\\
\hline
\sphinxAtStartPar
\(S_t\)
&&
\sphinxAtStartPar
Solar time
\\
\hline
\sphinxAtStartPar
\(T^*\)
&
\sphinxAtStartPar
K
&
\sphinxAtStartPar
Scaling parameter of temperature in the boundary layer
\\
\hline
\sphinxAtStartPar
\(T_a\)
&
\sphinxAtStartPar
˚C
&
\sphinxAtStartPar
Air temperature in height \(z\)
\\
\hline
\sphinxAtStartPar
\(T_{a\_K}\)
&
\sphinxAtStartPar
K
&
\sphinxAtStartPar
Air temperature in height \(z\)
\\
\hline
\sphinxAtStartPar
\(T_B\)
&
\sphinxAtStartPar
K
&
\sphinxAtStartPar
Surface brightness temperature (\({\varepsilon} = 1.0\))
\\
\hline
\sphinxAtStartPar
\(T_{max}\)
&
\sphinxAtStartPar
˚C
&
\sphinxAtStartPar
Max. surface temperature in the image
\\
\hline
\sphinxAtStartPar
\(T_s\)
&
\sphinxAtStartPar
˚C
&
\sphinxAtStartPar
Surface temperature
\\
\hline
\sphinxAtStartPar
\(T_{s\_K}\)
&
\sphinxAtStartPar
K
&
\sphinxAtStartPar
Surface temperature
\\
\hline
\sphinxAtStartPar
\(T_{st}\)
&
\sphinxAtStartPar
°C
&
\sphinxAtStartPar
Air temperature measured at meteostation in height \(z_{st}\)
\\
\hline
\sphinxAtStartPar
\(T_{s\_dry}\)
&
\sphinxAtStartPar
K
&
\sphinxAtStartPar
Calculated maximal surface temperature
\\
\hline
\sphinxAtStartPar
\(T_{s\_wet}\)
&
\sphinxAtStartPar
K
&
\sphinxAtStartPar
Calculated minimal surface temperature
\\
\hline
\sphinxAtStartPar
\(U\)
&
\sphinxAtStartPar
\(m.s^{-1}\)
&
\sphinxAtStartPar
Wind speed in height \(z\)
\\
\hline
\sphinxAtStartPar
\(u^*\)
&
\sphinxAtStartPar
\(m.s^{-1}\)
&
\sphinxAtStartPar
Friction velocity of the wind
\\
\hline
\sphinxAtStartPar
\(U_{st}\)
&
\sphinxAtStartPar
\(m.s^{-1}\)
&
\sphinxAtStartPar
Wind speed measured at meteostation in height \(z_{st}\)
\\
\hline
\sphinxAtStartPar
\(VPD\)
&
\sphinxAtStartPar
kPa
&
\sphinxAtStartPar
Water vapour pressure deficit
\\
\hline
\sphinxAtStartPar
\(W\)
&
\sphinxAtStartPar
mm
&
\sphinxAtStartPar
Amount of water for rain in atmosphere
\\
\hline
\sphinxAtStartPar
\(w_b\)
&
\sphinxAtStartPar
const.
&
\sphinxAtStartPar
Constants for spectral bands
\\
\hline
\sphinxAtStartPar
\(x\)
&
\sphinxAtStartPar
unitless
&
\sphinxAtStartPar
Constant
\\
\hline
\sphinxAtStartPar
\(z\)
&
\sphinxAtStartPar
m
&
\sphinxAtStartPar
Height of the mixing layer (\(z=200\))
\\
\hline
\sphinxAtStartPar
\(z_{0h}\)
&
\sphinxAtStartPar
m
&
\sphinxAtStartPar
Aerodynamic roughness for heat and water vapour transfer
\\
\hline
\sphinxAtStartPar
:math:{\color{red}\bfseries{}\textasciigrave{}}z\_\{0m\}
&
\sphinxAtStartPar
m
&
\sphinxAtStartPar
Aerodynamic roughness for momentum transfer
\\
\hline
\sphinxAtStartPar
\(z_{st}\)
&
\sphinxAtStartPar
m
&
\sphinxAtStartPar
Height of measurement at the meteostation
\\
\hline
\sphinxAtStartPar
\(\alpha\)
&
\sphinxAtStartPar
rel.
&
\sphinxAtStartPar
Albedo
\\
\hline
\sphinxAtStartPar
\(\alpha_{PT}\)
&
\sphinxAtStartPar
unitless
&
\sphinxAtStartPar
Priestley\sphinxhyphen{}Taylor alpha (\(\alpha_{PT}=1.26\))
\\
\hline
\sphinxAtStartPar
\(\alpha_z\)
&
\sphinxAtStartPar
°
&
\sphinxAtStartPar
Solar height angle
\\
\hline
\sphinxAtStartPar
\(\beta\)
&
\sphinxAtStartPar
unitless
&
\sphinxAtStartPar
Bowen ratio
\\
\hline
\sphinxAtStartPar
\(\beta_s\)
&
\sphinxAtStartPar
°
&
\sphinxAtStartPar
Slope gradient
\\
\hline
\sphinxAtStartPar
\(\Gamma\)
&
\sphinxAtStartPar
\(°C.m^{-1}\)
&
\sphinxAtStartPar
Adiabatic lapse rate (\({\Gamma=0.0065}\ °C.m^{-1}\))
\\
\hline
\sphinxAtStartPar
\(\gamma\)
&
\sphinxAtStartPar
\(kPa.˚C^{-1}\)
&
\sphinxAtStartPar
Psychrometric constant
\\
\hline
\sphinxAtStartPar
\(\gamma^*\)
&
\sphinxAtStartPar
\(kPa.˚C^{-1}\)
&
\sphinxAtStartPar
Modified psychrometric constant according to Jackson et al. (1981)
\\
\hline
\sphinxAtStartPar
\(\delta_s\)
&
\sphinxAtStartPar
°
&
\sphinxAtStartPar
Solar declination
\\
\hline
\sphinxAtStartPar
\(\delta T\)
&
\sphinxAtStartPar
K
&
\sphinxAtStartPar
Temperature gradient calculated according to Bastiaanssen et al. (1998)
\\
\hline
\sphinxAtStartPar
\(\delta T_{dry}\)
&
\sphinxAtStartPar
K
&
\sphinxAtStartPar
Temperature gradient for the dry areas
\\
\hline
\sphinxAtStartPar
\(\Delta\)
&
\sphinxAtStartPar
\(kPa.˚C^{-1}\)
&
\sphinxAtStartPar
The slope of the saturated water vapour pressure\textbackslash{} to temperature gradient
\\
\hline
\sphinxAtStartPar
\(\varepsilon\)
&
\sphinxAtStartPar
rel.
&
\sphinxAtStartPar
Surface emissivity
\\
\hline
\sphinxAtStartPar
\(\varepsilon_a\)
&
\sphinxAtStartPar
rel.
&
\sphinxAtStartPar
Atmospheric emissivity
\\
\hline
\sphinxAtStartPar
\(\varepsilon_s\)
&
\sphinxAtStartPar
rel.
&
\sphinxAtStartPar
Soil emissivity
\\
\hline
\sphinxAtStartPar
\(\varepsilon_v\)
&
\sphinxAtStartPar
rel.
&
\sphinxAtStartPar
Vegetation emissivity
\\
\hline
\sphinxAtStartPar
\(\kappa\)
&
\sphinxAtStartPar
unitless
&
\sphinxAtStartPar
Kármán constant (\(\kappa=0.41\))
\\
\hline
\sphinxAtStartPar
\(\lambda\)
&
\sphinxAtStartPar
\(J.g^{-1}\)
&
\sphinxAtStartPar
Latent heat of evaporation
\\
\hline
\sphinxAtStartPar
\(\lambda E\)
&
\sphinxAtStartPar
\(W.m^{-2}\)
&
\sphinxAtStartPar
Latent heat flux
\\
\hline
\sphinxAtStartPar
\(\lambda E_{max}\)
&
\sphinxAtStartPar
\(W.m^{-2}\)
&
\sphinxAtStartPar
Latent heat flux equal to \(Rn-G\)
\\
\hline
\sphinxAtStartPar
\(\lambda E_p\)
&
\sphinxAtStartPar
\(W.m^{-2}\)
&
\sphinxAtStartPar
Potential latent heat flux
\\
\hline
\sphinxAtStartPar
\(\lambda E_{PT}\)
&
\sphinxAtStartPar
\(W.m^{-2}\)
&
\sphinxAtStartPar
Priestley\sphinxhyphen{}Taylor potential latent heat flux
\\
\hline
\sphinxAtStartPar
\(\pi\)
&
\sphinxAtStartPar
unitless
&
\sphinxAtStartPar
Ludolf number
\\
\hline
\sphinxAtStartPar
\(\rho\)
&
\sphinxAtStartPar
\(kg.m^{-3}\)
&
\sphinxAtStartPar
Dry air density
\\
\hline
\sphinxAtStartPar
\(\rho_{s\_b}\)
&
\sphinxAtStartPar
rel.
&
\sphinxAtStartPar
Surface spectral reflectance for optical bands
\\
\hline
\sphinxAtStartPar
\(\rho_{t\_b}\)
&
\sphinxAtStartPar
rel.
&
\sphinxAtStartPar
TOA spectral reflectance for optical bands
\\
\hline
\sphinxAtStartPar
\(\varsigma\)
&
\sphinxAtStartPar
unitless
&
\sphinxAtStartPar
Monin\sphinxhyphen{}Obukhov stability parameter
\\
\hline
\sphinxAtStartPar
\(\sigma\)
&
\sphinxAtStartPar
\(W.m^{-2}.K^{-4}\)
&
\sphinxAtStartPar
Stefan\sphinxhyphen{}Boltzmann constant (\(\sigma=5.6703\cdot 10^{-8}\ W.m^{-2}.K^{-4}\))
\\
\hline
\sphinxAtStartPar
\(\Psi_h {(\varsigma)}\)
&
\sphinxAtStartPar
unitless
&
\sphinxAtStartPar
Stability parameter for heat transfer
\\
\hline
\sphinxAtStartPar
\(\Psi_m {(\varsigma)}\)
&
\sphinxAtStartPar
unitless
&
\sphinxAtStartPar
Stability parameter for momentum transfer
\\
\hline
\sphinxAtStartPar
\(\Omega\)
&
\sphinxAtStartPar
rel.
&
\sphinxAtStartPar
Decoupling coefficient (Omega factor)
\\
\hline
\sphinxAtStartPar
\(\eta\)
&
\sphinxAtStartPar
°
&
\sphinxAtStartPar
Satellite inclination angle to nadir
\\
\hline
\sphinxAtStartPar
\(\theta\)
&
\sphinxAtStartPar
°
&
\sphinxAtStartPar
Solar zenith angle
\\
\hline
\sphinxAtStartPar
\(\tau_{in\_b}\)
&
\sphinxAtStartPar
rel.
&
\sphinxAtStartPar
Atmospheric transmittance for spectral bands for direct radiation
\\
\hline
\sphinxAtStartPar
\(\tau_{out\_b}\)
&
\sphinxAtStartPar
rel.
&
\sphinxAtStartPar
Atmospheric transmittance for spectral bands for diffuse radiation
\\
\hline
\end{longtable}\sphinxatlongtableend\end{savenotes}



\renewcommand{\indexname}{Index}
\printindex
\end{document}